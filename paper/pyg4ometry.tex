%% This template can be used to write a paper for
%% Computer Physics Communications using LaTeX.
%% For authors who want to write a computer program description,
%% an example Program Summary is included that only has to be
%% completed and which will give the correct layout in the
%% preprint and the journal.
%% The `elsarticle' style is used and more information on this style
%% can be found at 
%% http://www.elsevier.com/wps/find/authorsview.authors/elsarticle.
%%
%%
%% \documentclass[preprint,12pt]{elsarticle}

%% Use the option review to obtain double line spacing
%% \documentclass[preprint,review,12pt]{elsarticle}

%% Use the options 1p,twocolumn; 3p; 3p,twocolumn; 5p; or 5p,twocolumn
%% for a journal layout:
%% \documentclass[final,1p,times]{elsarticle}
%% \documentclass[final,1p,times,twocolumn]{elsarticle}
%% \documentclass[final,3p,times]{elsarticle}
%\documentclass[final,3p,times,twocolumn]{elsarticle}
%% \documentclass[final,5p,times]{elsarticle}
\documentclass[final,5p,times,twocolumn]{elsarticle}

%% if you use PostScript figures in your article
%% use the graphics package for simple commands
%% \usepackage{graphics}
%% or use the graphicx package for more complicated commands
%% \usepackage{graphicx}
%% or use the epsfig package if you prefer to use the old commands
%% \usepackage{epsfig}

%% The amssymb package provides various useful mathematical symbols
\usepackage{amssymb}
\usepackage{fancyvrb}
%% The amsthm package provides extended theorem environments
%% \usepackage{amsthm}

%% The lineno packages adds line numbers. Start line numbering with
%% \begin{linenumbers}, end it with \end{linenumbers}. Or switch it on
%% for the whole article with \linenumbers after \end{frontmatter}.
%% \usepackage{lineno}

%% natbib.sty is loaded by default. However, natbib options can be
%% provided with \biboptions{...} command. Following options are
%% valid:

%%   round  -  round parentheses are used (default)
%%   square -  square brackets are used   [option]
%%   curly  -  curly braces are used      {option}
%%   angle  -  angle brackets are used    <option>
%%   semicolon  -  multiple citations separated by semi-colon
%%   colon  - same as semicolon, an earlier confusion
%%   comma  -  separated by comma
%%   numbers-  selects numerical citations
%%   super  -  numerical citations as superscripts
%%   sort   -  sorts multiple citations according to order in ref. list
%%   sort&compress   -  like sort, but also compresses numerical citations
%%   compress - compres ses without sorting
%%
%% \biboptions{comma,round}

% \biboptions{}

%% This list environment is used for the references in the
%% Program Summary
%%
\newcounter{bla}
\newenvironment{refnummer}{%
\list{[\arabic{bla}]}%
{\usecounter{bla}%
 \setlength{\itemindent}{0pt}%
 \setlength{\topsep}{0pt}%
 \setlength{\itemsep}{0pt}%
 \setlength{\labelsep}{2pt}%
 \setlength{\listparindent}{0pt}%
 \settowidth{\labelwidth}{[9]}%
 \setlength{\leftmargin}{\labelwidth}%
 \addtolength{\leftmargin}{\labelsep}%
 \setlength{\rightmargin}{0pt}}}
 {\endlist}

\journal{Computer Physics Communications}

\begin{document}

\begin{frontmatter}

%% Title, authors and addresses

%% use the tnoteref command within \title for footnotes;
%% use the tnotetext command for the associated footnote;
%% use the fnref command within \author or \address for footnotes;
%% use the fntext command for the associated footnote;
%% use the corref command within \author for corresponding author footnotes;
%% use the cortext command for the associated footnote;
%% use the ead command for the email address,
%% and the form \ead[url] for the home page:
%%
%% \title{Title\tnoteref{label1}}
%% \tnotetext[label1]{}
%% \author{Name\corref{cor1}\fnref{label2}}
%% \ead{email address}
%% \ead[url]{home page}
%% \fntext[label2]{}
%% \cortext[cor1]{}
%% \address{Address\fnref{label3}}
%% \fntext[label3]{}

\title{pyg4ometry: a python library for the creation of Monte Carlo particle transport geometries}

%% use optional labels to link authors explicitly to addresses:
%% \author[label1,label2]{<author name>}
%% \address[label1]{<address>}
%% \address[label2]{<address>}

\author[a]{Stewart T. Boogert\corref{author}}
\author[a]{Andrey Abramov}
\author[a]{Laurence Nevay}
\author[a]{William Shields}

\cortext[author] {Corresponding author.\\\textit{E-mail address:} stewart.boogert@rhul.ac.uk}
\address[a]{John Adams Institute at Royal Holloway, Department of Physics, Royal Holloway, Egham, TW20 0EX, Surrey, UK}

\begin{abstract}
\end{abstract}

\begin{keyword}
%% keywords here, in the form: keyword \sep keyword
Geant; Monte Carlo; Particle; Transport; Geometry; 

\end{keyword}

\end{frontmatter}

%%
%% Start line numbering here if you want
%%
% \linenumbers

% Computer program descriptions should contain the following
% PROGRAM SUMMARY.

{\bf PROGRAM SUMMARY/NEW VERSION PROGRAM SUMMARY}
  %Delete as appropriate.

\begin{small}
\noindent
{\em Program Title: pyg4ometry }                                         \\
{\em Licensing provisions(please choose one): GPLv3 }                                   \\
{\em Programming language: python, C/C++}                                   \\

{\em Supplementary material:}                                 \\
  % Fill in if necessary, otherwise leave out.
{\em Journal reference of previous version:}                  \\
  %Only required for a New Version summary, otherwise leave out.
{\em Does the new version supersede the previous version?:}   \\
  %Only required for a New Version summary, otherwise leave out.
{\em Reasons for the new version:}\\
  %Only required for a New Version summary, otherwise leave out.
{\em Summary of revisions:}*\\
  %Only required for a New Version summary, otherwise leave out.

{\em Nature of problem(approx. 50-250 words):}\\
Creating computer readable geometry descriptions for Monte Carlo particle transport codes is a time consuming and error prone task.  \\
{\em Solution method(approx. 50-250 words):}\\
  %Describe the method solution here.
{\em Additional comments including Restrictions and Unusual features (approx. 50-250 words):}\\
  %Provide any additional comments here.
   \\

\begin{thebibliography}{0}
\bibitem{1}Reference 1         % This list should only contain those items referenced in the                 
\bibitem{2}Reference 2         % Program Summary section.   
\bibitem{3}Reference 3         % Type references in text as [1], [2], etc.
                               % This list is different from the bibliography at the end of 
                               % the Long Write-Up.
\end{thebibliography}
* Items marked with an asterisk are only required for new versions
of programs previously published in the CPC Program Library.\\
\end{small}


%% main text
\section{Introduction} \label{sec:introduction}
There are numerous different software codes to simulate the passage of particles through material, so called particle transport (PT) programmes, these 
include MCMP, FLUKA, Geant3 and Geant4. Each code either has a particular specialism or target user community.  PT Monte Carlo simulations have 
diverse uses, shielding calculations for radiological protection, detector performance, medical imaging and therapy, space radiation environment simulation 
are some examples. A fundamental requirement of all of the codes is to supply a computer readable description of the physical geometry that the particles 
are passing through.  The creation of geometry files is typically a very time consuming activity, the simulation validity and performance is directly dependent 
on the quality of the geometry. The requirement of a  geometry system that allows the conversion between files prepared for different codes will enable cross 
checks of the physics processes in particle transport codes. Parallel to the creation of geometry files for PT programs, usually computer assisted design (CAD) 
files exist for systems which need to be simulated in PT programmes. The fundamental geometric representations in CAD files is typically not amenable for PT 
programmes.  For these reasons it is advantageous to create a software tool that allows particle transport code users to rapidly develop geometry files, convert between 
common PT programme geometry formats and finally import CAD models. This paper describes such geometry creation and conversion package 
called {\em pyg4ometry}, implemented in {\em Python} and internally based on the Geant4 geometry description. The main features of  {\em pyg4ometry} 
are a python scripting API to rapidly design parametrised geometry, conversion to and from  FLUKA geometry descriptions, conversion from CAD formats 
(STEP and IGES) based on FreeCAD-OpenCascade and powerful geometry visualisation tools based on VTK. 

\section{Geometric descriptions} \label{sec:geometric}
Central to description of a geometric description of a system is how a solid is defined in three dimensions. There are numerous different ways to describe a
geometry and these are described briefly in the following sections.  

\subsection{Constructive solid geometry (CSG)}
Constructive solid geometry uses boolean operations to model complex surfaces which represent a solid. Typically simple finite or infinite solids are combined 
with operations, such as union, intersection and difference to model more complex solids. FLUKA uses CSG to model solids.

\subsection{Boundary representation (BREP)}
Boundary representation consists of two parts, topology and geometry. Topological elements are faces, edges and vertices. No current PT applications use native 
file formats employed by CAD systems. The conversion of CAD BREP formats for loading in PT applications is typically performed via a tessellated format. 

\subsection{Tessellated polygons}
Solid volumes can be defined using triangular, quadrilateral or tetrahedral meshes. Numerous formats exist to describe meshes, the ubiquitous being STL and more 
modern variants being PLY, OBJ, etc.  

\subsection{Geant4 geometery}
Geant4 geometry description is a mixture of BREP, CSG and tessellated concepts. Geant4 includes XX fundamental solids, although it does not store a sense 
of topology present in traditional CAD BREP systems. One of the fundamental solids is a {\em TessellatedSolid} which can used to represent STL or PLY files. 
Geant4 also provides the ability to perform boolean operations on these primitive solids. The richest and most flexible geometry description is currently used by
Geant4. Not only do solid objects need to be defined but also placed in a world coordinate system. Geant4 has two concepts which facilitate this {\it Logical Volumes} 
and {\it Physical Volumes}. A Logical Volume is a volume that has an outer solid but also material, magnetic field and zero or more physical {\it daughter} volumes. A Physical 
Volume is a placement (or instance) of a Logical Volume.   

\subsection{Geant Detector Mark-up Language} 
To exchange geometry descriptions Geant Detector Mark-up Language (GDML) was developed. Geant4 and ROOT can read and write GDML format and it is commonly
used as an exchange format for Geant4 geometries. 
\begin{table}[hbt!]
\centering
\begin{tabular}{|c|c|c|} \hline
GDML tag 		&  Child tags				& pyg4ometry class 			\\ \hline
Defines 			&						&						\\
Solids 			&						&						\\
Materials 			&						&						\\
Structure			&						&						\\ \hline
\end{tabular}
\caption{GDML tags and corresponding {\it pyg4ometry} classes.}
\end{table}

\section{{\em pyg4ometry} design and layout}
The core of {\em pyg4ometry} consists of python classes that mimic Geant4 solids, logical, physical volume objects, GDML parameters and material classes.
The constructors of the python classes are kept as close to the original Geant4 C++ implementation so that {\it pyg4ometry} users do not have to learn 
a new API. Geometry construction in python proceeds in a way which is very similar to geometry construction in Geant4. A user relatively familiar with Geant4
should be able to start creating geometry in pyg4ometry immediately. In the following sections novel or important developments in {\it pyg4ometry} are described.  
    
\subsection{Expression engine}
In GDML symbolic expressions can be used to parametrise solids and their placement. These expressions are evaluated when the GDML is loaded into Geant4. 
In order to fully replicate the functionality of GDML an expression engine was implemented using ANTLR. GDML allows for the definition and assignment of 
variables. GDML expressions are not much more complicated than binary operators $+, -, \times, /$ and common trigonometric and special functions $\sin, \cos, \tan$ etc. 
The AST terminates on either expressions which evaluate to numbers or GDML variables. Internally all classes use GDML expressions and not floating point numbers. 
Storing internal data as expressions allows for deferred evaluation (or re-evaluation) of solid parameters and placements. This allows a user to update variables whilst
defining geometry and the expression engine will update all internal values.  
 
\subsection{Tessellation engine}
For each Geant4 primitive solid a triangular tessellated mesh is generated and cached. This mesh is then used to determine the extent of placed instances of 
geometry (physical volumes) and meshes for CSG derived solids. CSG mesh calculations are performed using a BSP tree technique. Triangular meshes based
on CSG operations involving curved surfaces often contain large numbers of thin triangles. Before meshes are visualised or written to disk various algorithms from 
TetGEN can be employed to give the meshes more desirable output. 

\subsection{Internal data representation} 
The internal data representation follows closely the structure of GDML. A registry class aggregates python ordered dictionaries that are  used to store the main 
elements of a GDML file. As a user specifies geometry the registry is updated. When a user is finished with the geometry, the registry is written to disk as a GDML file.  
  
\subsection{Input and output} 
For each input format available in {\it pyg4ometry} a dedicated {\it Reader} class is implemented, so \verb|gdml.Reader|, \verb|stl.Reader|, \verb|fluka.Reader| 
and \verb|freecad.Reader|. Each reader constructs the appropriate Geant4 classes and provides a registry which can be used or manipulated by the user. 
Output consists of taking the registry and writing the appropriate GDML.

\subsection{Overlap detection}
All PT transport codes can not handle spatial overlap between two geometric objects and likely are to have ill defined behaviour when encountering such a 
situation.  A key feature of {\it pyg4ometry} is the detection of potential overlaps efficiently, it does this by performing a CSG intersection between solid instances 
and determining if the resulting mesh is empty. 
  
\subsection{Visualisation}
A  geometry hierarchy can be viewed using the popular Visualisation Toolkit (VTK). No separate scene graph is required as the Geant4 volume hierarchy is sufficient 
to place the meshes associated with each physical volume. The python version of Geant4's physical volume is also used to store visualisation attributes like the solids 
color, surface or wireframe representation and visibility. Overlaps detected in the discretised mesh geometry can be displayed separately to allow a user to remove
the potential overlaps. This is dependent on the accuracy of the mesh generated and is a particular problem with meshes based on curved surfaces.    

\section{Rapid geometry modelling}
Given the python scripting interface, expression  and tessellation engines. It is possible for a user to rapidly specify the geometrical layout of the PT problem, vary 
the parameters of the geometry and visualise the geometry. Finally when the desired result without geometry overlaps is finalised a GDML file can be output
from the internal memory representation. 

{\small
\begin{verbatim}
import pyg4ometry.gdml as gd
import pyg4ometry.geant4 as g4
import pyg4ometry.visualisation as vi

# create empty data storage structure
reg = g4.Registry()

# expressions 
wx = gd.Constant("wx","100",reg)
wy = gd.Constant("wy","100",reg)
wz = gd.Constant("wz","100",reg)

bx = gd.Constant("wx","10",reg)
by = gd.Constant("wy","10",reg)
bz = gd.Constant("wz","10",reg)

# materials
bm = g4.Material("G4_Galactic") 
wm = g4.Material("G4_Iron") 

# solids
wb = g4.solid.Box("wb",wx,wy,wz,reg)
b  = g4.solid.Box("b",bx,by,bz)

# structure 
wl = g4.LogicalVolume(wb, wm, "wl", reg)
bl = g4.LogicalVolume(b, bm, "b", reg)
bp1 = g4.PhysicalVolume([0,0,0],[0,0,0], 
                        bl, "b_pv1", wl) 
bp2 = g4.PhysicalVolume([0,0,0],[-2*bx,0,0], 
                        bl, "b_pv2", wl)  
bp3 = g4.PhysicalVolume([0,0,0],[2*bx,0,0], 
                        bl, "b_pv3", wl) 
                        
# gdml output
w = gd.Writer()
w.write(reg,"output.gdml")

# visualisation 
v = vi.VtkViewer()
v.addLogicalVolume(wl)
v.view()
\end{verbatim}
}

\section{CAD to Geant4 conversion }
STEP and IGES files can be loaded into pyg4ometry, via an interface based on FreeCAD. FreeCAD is an open source CAD/CAM program, which in turn 
is based on OpenCASCADE. A STEP CAD file could be considered as a hierarchical tree of {\it part assemblies} and {\it part features}, where a part assembly is a 
collection of part features.   

\section{Geant4 and FLUKA conversion}
FLUKA geometry is based upon a limited set of primitive volumes which can be combined using logical operations. Many FLUKA primitives are infinite in extent, so 
to emulate FLUKA geometry construction large but finite extent Geant4 primitive solids are used. Table XXX shows the mapping between FLUKA and Geant4 objects. 

\section{Performance of converted geometry in Geant4}
Clearly for Geant4 geometry defined in {\it pyg4ometry} the performance is exactly the same as it the geometry was specified in C++ and linked against the Geant4
libraries. A comparison of the performance of converted CAD or FLUKA geometry compared with native Geant4 solids is required so a user can optimise the available
computing resources against the human resources (required to convert geometry).

\section{Conclusions}
The authors believe that tools to quickly create, either ab-initio or by conversion geometry usable in particle transport Monte Carlo programmes will save significant 
amounts of user effort and ultimately yield more accurate simulations. {\it pyg4ometry} is a relatively complete implementation of a geometry creation tool, whilst heavily 
internally based on Geant4 and GDML can have utility for users of all PT codes. {\it pyg4ometry} can clearly be extended to other formats or applications. As for applications,
examples include  the  generation triangular mesh structures for GPU accelerated photon tracking in liquid noble  dark matter detectors and geometries for event visualisation. 
Although not implemented in the current version, a C++ output writer can be quickly implemented to generate geometry which can be compiled into a Geant4 application. 

%% The Appendices part is started with the command \appendix;
%% appendix sections are then done as normal sections
%% \appendix

%% \section{}
%% \label{}

%% References
%%
%% Following citation commands can be used in the body text:
%% Usage of \cite is as follows:
%%   \cite{key}         ==>>  [#]
%%   \cite[chap. 2]{key} ==>> [#, chap. 2]
%%

%% References with bibTeX database:

\bibliographystyle{elsarticle-num}
\bibliography{<your-bib-database>}

%% Authors are advised to submit their bibtex database files. They are
%% requested to list a bibtex style file in the manuscript if they do
%% not want to use elsarticle-num.bst.

%% References without bibTeX database:

% \begin{thebibliography}{00}

%% \bibitem must have the following form:
%%   \bibitem{key}...
%%

% \bibitem{}

% \end{thebibliography}


\end{document}

%%
%% End of file 