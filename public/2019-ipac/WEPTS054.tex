% !TeX spellcheck = en_GB
% !TeX program = lualatex
%
% v 2.3  Feb 2019   Volker RW Schaa
%		# changes in the collaboration therefore updated file "jacow-collaboration.tex"
%		# all References with DOIs have their period/full stop before the DOI (after pp. or year)
%		# in the author/affiliation block all ZIP codes in square brackets removed as it was not %         understood as optional parameter and ZIP codes had bin put in brackets
%       # References to the current IPAC are changed to "IPAC'19, Melbourne, Australia"
%       # font for ‘url’ style changed to ‘newtxtt’ as it is easier to distinguish "O" and "0"
%
\documentclass[a4paper,
               %boxit,        % check whether paper is inside correct margins
               %titlepage,    % separate title page
               %refpage       % separate references
               %biblatex,     % biblatex is used
               keeplastbox,   % flushend option: not to un-indent last line in References
               %nospread,     % flushend option: do not fill with whitespace to balance columns
               %hyphens,      % allow \url to hyphenate at "-" (hyphens)
               %xetex,        % use XeLaTeX to process the file
               %luatex,       % use LuaLaTeX to process the file
               ]{jacow}
%
% ONLY FOR \footnote in table/tabular
%
\usepackage{pdfpages,multirow,ragged2e} %
\usepackage{listings}
\usepackage{fancyvrb}

\lstset{
 identifierstyle=\small, 
 commentstyle=\color{red}, 
 stringstyle=\ttfamily, 
 showstringspaces=false
 }


%
% CHANGE SEQUENCE OF GRAPHICS EXTENSION TO BE EMBEDDED
% ----------------------------------------------------
% test for XeTeX where the sequence is by default eps-> pdf, jpg, png, pdf, ...
%    and the JACoW template provides JACpic2v3.eps and JACpic2v3.jpg which
%    might generates errors, therefore PNG and JPG first
%
\makeatletter%
	\ifboolexpr{bool{xetex}}
	 {\renewcommand{\Gin@extensions}{.pdf,%
	                    .png,.jpg,.bmp,.pict,.tif,.psd,.mac,.sga,.tga,.gif,%
	                    .eps,.ps,%
	                    }}{}
\makeatother

% CHECK FOR XeTeX/LuaTeX BEFORE DEFINING AN INPUT ENCODING
% --------------------------------------------------------
%   utf8  is default for XeTeX/LuaTeX
%   utf8  in LaTeX only realises a small portion of codes
%
\ifboolexpr{bool{xetex} or bool{luatex}} % test for XeTeX/LuaTeX
 {}                                      % input encoding is utf8 by default
 {\usepackage[utf8]{inputenc}}           % switch to utf8

\usepackage[USenglish]{babel}

%
% if BibLaTeX is used
%
\ifboolexpr{bool{jacowbiblatex}}%
 {%
  \addbibresource{jacow-test.bib}
  \addbibresource{biblatex-examples.bib}
 }{}
\listfiles

%%
%%   Lengths for the spaces in the title
%%   \setlength\titleblockstartskip{..}  %before title, default 3pt
%%   \setlength\titleblockmiddleskip{..} %between title + author, default 1em
%%   \setlength\titleblockendskip{..}    %afterauthor, default 1em

\begin{document}

\title{Pyg4ometry : A Tool to Create Geometries for Geant4, BDSIM, G4Beamline and FLUKA for Particle Loss and Energy Deposit Studies}

\author{Stewart Boogert\thanks{stewart.boogert@rhul.ac.uk}, Andrey Abramov, Joshua Albrecht, \\ Gian Luigi D'Alessandro, Laurence Nevay, William Shields, Stuart Walker \\
JAI, Egham, Surrey)}
	
\maketitle

%
\begin{abstract}
Studying the energy deposits in accelerator components, mechanical supports, services, ancillary equipment and shielding requires a detailed computer readable description of the component geometry. The creation of geometries is a significant bottle neck in producing complete simulation models and reducing the effort required will provide the ability of non- experts to simulate the effects of beam losses on realistic accelerators. The paper describes a flexible and easy to use PYTHON package to create geometries usable by either Geant4 (and so BDSIM or G4Beamline) or FLUKA either from scratch or by conversion from common engineering formats, such as STEP or IGES created by industry standard CAD/CAM packages. The conversion requires an intermediate conversion to STL or similar triangular or tetrahedral tessellation description. A key capability of \verb|pyg4ometry| is to mix GDML/STEP/STL geometries and visualisation of the resulting geometry and determine if there are any geometric overlaps. An example conversion of a complex geometry is presented along with performance results for physical particle tracking of electromagnetically and hadronically produced particles using BDSIM.
\end{abstract}


\section{INTRODUCTION}

\subsection{Particle transport codes}
There are multiple different codes to simulate the transportation and physics processes of particles though accelerators and detectors, these include Geant4 \cite{geant4}, MCMPX and FLUKA \cite{fluka}. 

\subsection{Geometry generation}

\section{EXAMPLES}
\subsection{Python}

\VerbatimInput[fontsize=\small]{./examples/simple.py}

\begin{figure}[!htb]
   \centering
   \includegraphics*[width=.9\columnwidth]{./examples/simple.jpg}
   \caption{Example of three primitive Geant4 solids rendered in VTK.}
   \label{fig:simple}
\end{figure}

Figure \ref{fig:dipole} shows a more complex geometry example. 
\begin{figure}[!htb]
   \centering
   \includegraphics*[width=.9\columnwidth]{./examples/dipole.jpg}
   \caption{Example of a more complex geometry example, this time a cavity beam position monitor}
   \label{fig:dipole}
\end{figure}


%\lstinputlisting[language=Python]{./examples/simple.py}

\subsection{Geometry Description Markup Language (GDML)}

\subsection{Standard tessellation language (STL)}

\subsection{Computer aided design (STEP)}


\section{SUMMARY}

\section{ACKNOWLEDGEMENTS}
Any acknowledgement should be in a separate section directly preceding
the \textbf{REFERENCES} or \textbf{APPENDIX} section.


%
% only for "biblatex"
%

\ifboolexpr{bool{jacowbiblatex}}%
	{\printbibliography}%
	{%
	% "biblatex" is not used, go the "manual" way
	
	%\begin{thebibliography}{99}   % Use for  10-99  references
	\begin{thebibliography}{9} % Use for 1-9 references
	
	\bibitem{geant4} Recent developments in Geant4, NIMA 835, pages 186-225, 2016 
	\bibitem{fluka} CERN-2005-10 (2005), 
	

	\end{thebibliography}
} % end \ifboolexpr


% for use as JACoW template the inclusion of the ANNEX parts have been commented out
% to generate the complete documentation please remove the "%" of the next two commands
% 
%%%\newpage

%@article{ALLISON2016186,
%title = "Recent developments in Geant4",
%journal = "Nuclear Instruments and Methods in Physics Research Section A: Accelerators, Spectrometers, Detectors and Associated Equipment",
%volume = "835",
%pages = "186 - 225",
%year = "2016",
%issn = "0168-9002",
%doi = "https://doi.org/10.1016/j.nima.2016.06.125",
%url = "http://www.sciencedirect.com/science/article/pii/S0168900216306957",
%author = "J. Allison and K. Amako and J. Apostolakis and P. Arce and M. Asai and T. Aso and E. Bagli and A. Bagulya and S. Banerjee and G. Barrand and B.R. Beck and A.G. Bogdanov and D. Brandt and J.M.C. Brown and H. Burkhardt and Ph. Canal and D. Cano-Ott and S. Chauvie and K. Cho and G.A.P. Cirrone and G. Cooperman and M.A. Cortés-Giraldo and G. Cosmo and G. Cuttone and G. Depaola and L. Desorgher and X. Dong and A. Dotti and V.D. Elvira and G. Folger and Z. Francis and A. Galoyan and L. Garnier and M. Gayer and K.L. Genser and V.M. Grichine and S. Guatelli and P. Guèye and P. Gumplinger and A.S. Howard and I. Hřivnáčová and S. Hwang and S. Incerti and A. Ivanchenko and V.N. Ivanchenko and F.W. Jones and S.Y. Jun and P. Kaitaniemi and N. Karakatsanis and M. Karamitros and M. Kelsey and A. Kimura and T. Koi and H. Kurashige and A. Lechner and S.B. Lee and F. Longo and M. Maire and D. Mancusi and A. Mantero and E. Mendoza and B. Morgan and K. Murakami and T. Nikitina and L. Pandola and P. Paprocki and J. Perl and I. Petrović and M.G. Pia and W. Pokorski and J.M. Quesada and M. Raine and M.A. Reis and A. Ribon and A. Ristić Fira and F. Romano and G. Russo and G. Santin and T. Sasaki and D. Sawkey and J.I. Shin and I.I. Strakovsky and A. Taborda and S. Tanaka and B. Tomé and T. Toshito and H.N. Tran and P.R. Truscott and L. Urban and V. Uzhinsky and J.M. Verbeke and M. Verderi and B.L. Wendt and H. Wenzel and D.H. Wright and D.M. Wright and T. Yamashita and J. Yarba and H. Yoshida",
%keywords = "High energy physics, Nuclear physics, Radiation, Simulation, Computing",
%abstract = "Geant4 is a software toolkit for the simulation of the passage of particles through matter. It is used by a large number of experiments and projects in a variety of application domains, including high energy physics, astrophysics and space science, medical physics and radiation protection. Over the past several years, major changes have been made to the toolkit in order to accommodate the needs of these user communities, and to efficiently exploit the growth of computing power made available by advances in technology. The adaptation of Geant4 to multithreading, advances in physics, detector modeling and visualization, extensions to the toolkit, including biasing and reverse Monte Carlo, and tools for physics and release validation are discussed here."
%}

%"FLUKA: a multi-particle transport code"
%A. Ferrari, P.R. Sala, A. Fasso`, and J. Ranft,
% CERN-2005-10 (2005), INFN/TC_05/11, SLAC-R-773

%%%\include{annexes-A4}

\end{document}
