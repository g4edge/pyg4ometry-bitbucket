% !TeX spellcheck = en_GB
% !TeX program = lualatex
%
% v 2.3  Feb 2019   Volker RW Schaa
%		# changes in the collaboration therefore updated file "jacow-collaboration.tex"
%		# all References with DOIs have their period/full stop before the DOI (after pp. or year)
%		# in the author/affiliation block all ZIP codes in square brackets removed as it was not %         understood as optional parameter and ZIP codes had bin put in brackets
%       # References to the current IPAC are changed to "IPAC'19, Melbourne, Australia"
%       # font for ‘url’ style changed to ‘newtxtt’ as it is easier to distinguish "O" and "0"
%
\documentclass[a4paper,
               %boxit,        % check whether paper is inside correct margins
               %titlepage,    % separate title page
               %refpage       % separate references
               %biblatex,     % biblatex is used
               keeplastbox,   % flushend option: not to un-indent last line in References
               %nospread,     % flushend option: do not fill with whitespace to balance columns
               %hyphens,      % allow \url to hyphenate at "-" (hyphens)
               %xetex,        % use XeLaTeX to process the file
               %luatex,       % use LuaLaTeX to process the file
               ]{jacow}
%
% ONLY FOR \footnote in table/tabular
%
\usepackage{pdfpages,multirow,ragged2e} %
%
% CHANGE SEQUENCE OF GRAPHICS EXTENSION TO BE EMBEDDED
% ----------------------------------------------------
% test for XeTeX where the sequence is by default eps-> pdf, jpg, png, pdf, ...
%    and the JACoW template provides JACpic2v3.eps and JACpic2v3.jpg which
%    might generates errors, therefore PNG and JPG first
%
\makeatletter%
	\ifboolexpr{bool{xetex}}
	 {\renewcommand{\Gin@extensions}{.pdf,%
	                    .png,.jpg,.bmp,.pict,.tif,.psd,.mac,.sga,.tga,.gif,%
	                    .eps,.ps,%
	                    }}{}
\makeatother

% CHECK FOR XeTeX/LuaTeX BEFORE DEFINING AN INPUT ENCODING
% --------------------------------------------------------
%   utf8  is default for XeTeX/LuaTeX
%   utf8  in LaTeX only realises a small portion of codes
%
\ifboolexpr{bool{xetex} or bool{luatex}} % test for XeTeX/LuaTeX
 {}                                      % input encoding is utf8 by default
 {\usepackage[utf8]{inputenc}}           % switch to utf8

\usepackage[USenglish]{babel}

%
% if BibLaTeX is used
%
\ifboolexpr{bool{jacowbiblatex}}%
 {%
  \addbibresource{jacow-test.bib}
  \addbibresource{biblatex-examples.bib}
 }{}
\listfiles

%%
%%   Lengths for the spaces in the title
%%   \setlength\titleblockstartskip{..}  %before title, default 3pt
%%   \setlength\titleblockmiddleskip{..} %between title + author, default 1em
%%   \setlength\titleblockendskip{..}    %afterauthor, default 1em

\begin{document}

\title{Pyg4ometry : A Tool to Create Geometries for Geant4, BDSIM, G4Beamline and FLUKA for Particle Loss and Energy Deposit Studies}

\author{Stewart Boogert\thanks{stewart.boogert@rhul.ac.uk}, Andrey Abramov, Joshua Albrecht, \\ Gian Luigi D'Alessandro, Laurence Nevay, William Shields, Stuart Walker \\
JAI, Egham, Surrey)}
	
\maketitle

%
\begin{abstract}
Studying the energy deposits in accelerator components, mechanical supports, services, ancillary equipment and shielding requires a detailed computer readable description of the component geometry. The creation of geometries is a significant bottle neck in producing complete simulation models and reducing the effort required will provide the ability of non- experts to simulate the effects of beam losses on realistic accelerators. The paper describes a flexible and easy to use PYTHON package to create geometries usable by either Geant4 (and so BDSIM or G4Beamline) or FLUKA either from scratch or by conversion from common engineering formats, such as STEP or IGES created by industry standard CAD/CAM packages. The conversion requires an intermediate conversion to STL or similar triangular or tetrahedral tessellation description. A key capability of \verb|pyg4ometry| is to mix GDML/STEP/STL geometries and visualisation of the resulting geometry and determine if there are any geometric overlaps. An example conversion of a complex geometry is presented along with performance results for physical particle tracking of electromagnetically and hadronically produced particles using BDSIM.
\end{abstract}


\section{INTRODUCTION}

\subsection{Particle transport codes}

\subsection{Geometry generation}

\section{EXAMPLES}
\subsection{Python}

\subsection{Geometry Description Markup Language (GDML)}

\subsection{Computer aided design (STEP)}

\section{SUMMARY}

\section{ACKNOWLEDGEMENTS}
Any acknowledgement should be in a separate section directly preceding
the \textbf{REFERENCES} or \textbf{APPENDIX} section.


%
% only for "biblatex"
%
\ifboolexpr{bool{jacowbiblatex}}%
	{\printbibliography}%
	{%
	% "biblatex" is not used, go the "manual" way
	
	%\begin{thebibliography}{99}   % Use for  10-99  references
	\begin{thebibliography}{9} % Use for 1-9 references
	
	\bibitem{jacow-help}
		JACoW,
		\url{http://www.jacow.org}
	
	\bibitem{IEEE}
		\textit{IEEE Editorial Style Manual},
		IEEE Periodicals, Piscataway,
		NJ, USA, Oct. 2014, pp. 34--52.

	\bibitem{journal-abbreviations}
	\url{https://woodward.library.ubc.ca/researchhelp/journal-abbreviations/}

	\end{thebibliography}
} % end \ifboolexpr
%
% for use as JACoW template the inclusion of the ANNEX parts have been commented out
% to generate the complete documentation please remove the "%" of the next two commands
% 
%%%\newpage

%%%\include{annexes-A4}

\end{document}
